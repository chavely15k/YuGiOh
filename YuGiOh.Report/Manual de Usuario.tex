\documentclass[a4paper]{article}
\usepackage[margin=2cm]{geometry}
\usepackage{graphicx}
\usepackage{enumitem}
\usepackage[utf8]{inputenc}
\setlist[description]{}

\begin{document}

\title{Torneos de Yu-Gi-Oh}
\author{
  \begin{tabular}{c}
    Chavely 02110766835 \\
    Jos\'e Carlos CI \\
    L\'azaro David CI \\
    Max CI
  \end{tabular}
}
\date{\today}
\maketitle
\newpage

\section{Objetivos del producto}
Este sistema permitirá a los organizadores de torneos crear, administrar y seguir el progreso de torneos de
YuGiOh!, as\'i como proporcionar a los jugadores una plataforma para registrarse en estos eventos y consultar
estad\'isticas relevantes.

\newpage

\section{Requerimientos t\'ecnicos}

\newpage

\section{Requerimientos de software}
Tener instalados dotnet y npm en la computadora

\newpage

\section{Forma de instalar la aplicaci\'on}
\begin{enumerate}
\item Abrir una consola.
\item Colocarse en la carpeta ra\'iz del proyecto(YuGiOh) y navegar hacia ./YugiOh.API/.
\item escribir "dotnet run".
\end{enumerate}

\newpage

\section{Breve explicaci\'on de cada una de las opciones del sistema}

\begin{enumerate}
\item El sistema permite registrarse en la aplicaci\'on como usuario o como administrador insertando sus datos e incluyendo un codigo.
\item Los jugadores pueden ver los torneos disponibles y solicitar inscribirse en los mismos, as\'i como geationar la informaci\'on de sus decks.
\item Los administradores pueden crear torneos, ver las soicitudes pendientes de sus torneos y modificar la informaci\'on de los matches de sus torneos, as\'i como hacer todo lo que hace un jugador.
\end{enumerate}

\newpage

\section{Breve explicaci\'on de cada una de las salidas del sistema}
Debe contener imgs de tablas y graficas,  formatos empleados para exportar datos, etc... 


\end{document}
 
