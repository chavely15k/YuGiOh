\documentclass[a4paper]{article}
\usepackage[margin=2cm]{geometry}
\usepackage{graphicx}
\usepackage{enumitem}
\setlist[description]{}

\begin{document}

\title{Torneos de Yu-Gi-Oh}
\author{
  \begin{tabular}{c}
    Chavely Gonz\'alez Acosta C312 \\
    Jos\'e Carlos Pendas Rodrigu\'ez C311 \\
    L\'azaro David Alba Ajete C311 \\
    Max Bengochea Mor\'e C311
  \end{tabular}
}
\date{\today}
\maketitle
\newpage

\section{An\'alisis y reformulaci\'on enriquecedora de los requerimientos funcionales e informacionales del sistema}
\subsection{Requerimientos funcionales:}
\begin{enumerate}
	\item \textbf{Registro de usuarios:} Los usuarios se registran en la aplicación proporcionando su información personal: nombre completo, municipio, provincia, teléfono (opcional) y dirección.
	\item \textbf{Registro de Decks:} Los jugadores registraran sus decks (mazos de cartas) asoci\'andolos con su perfil, incluyendo detalles como: nombre del deck, cantidad de cartas en el mazo principal, en el mazo alternativo y en el mazo extra, así como su arquetipo.
	\item \textbf{Creaci\'on de torneos:} Los administradores pueden crear torneos especificando: nombre del torneo, fecha de inicio y dirección, un torneo puede ser creado por un solo administrador.
	\item \textbf{Solicitud de inscripci\'on en torneos:} Los jugadores solicitan inscribirse en los torneos disponibles, asegur\'andose de que no puedan inscribirse despu\'es de la fecha de inicio del torneo, y con esactamente un deck. 
	\item \textbf{Inscripci\'on en torneos:} Los administradores reciben las solicitudes de inscripci\'on de los jugadores a los torneos y se encargan de inscribirlos en los mismos.
    \item \textbf{Gestión de Rondas y Partidas:} La aplicaci\'on administra las rondas de los torneos, asignando aleatoriamente los emparejamientos.
    \item \textbf{Registro de los resultados de las partidas:} Los administradores registran los resultados de las partidas, asegurando que una partida de Yu-Gi-Oh! es de 3 a ganar 2.
    \item \textbf{Consultas y Estad\'isticas}: La aplicaci\'on ofrece una interfaz donde los usuarios pueden realizar consultas y obtener estad\'isticas, as\'i como exportar los resultados a otros formatos. Las consultas a modelarse son las siguientes:
    \begin{itemize}
    \item Los n jugadores con m\'as decks en su poder(de mayor a menor).
    \item Los n jugadores m\'as populares entre los jugadores(de mayor a menor).
    \item La provincia/municipio donde es m\'as popular un arquetipo.
    \item El campe\'on de un torneo.
    \item Los n jugadores con m\'as victorias (ordenados de mayor a menor).
    \item El arquetipo m\'as utilizado en un torneo dado.
    \item La cantidad de veces que los arquetipos han sido el arquetipo del campe\'on en un grupo de torneos(en un intervalo de tiempo).
    \item La provincia/municipio con m\'as campeones(en un intervalo de tiempo)
    \item Dado un torneo y una ronda, cu\'ales son los arquetipos m\'as representados(cantidad de jugadores us\'andolos)
    \item Los n arquetipos m\'as utilizados por al menos un jugador en el torneo(de mayor a menor)    
    \end{itemize}
    \item \textbf{Seguridad}: La aplicaci\'on implementa autenticaci\'on y autorizaci\'on para asegurarse de que solo los administradores pueden realizar acciones como crear torneos y registrar resultados de partidas. Adem\'as, se deben validar los datos de entrada para evitar errores o datos incorrectos en la base de datos.
\end{enumerate} 
\subsection{Requerimientos informacionales del sistema:}
Con el objetivo de validar las peticiones de los usuarios, as\'i como dar respuesta a las consultas de los mismos, es necesario mantener diversas informaciones en una base de datos:
\begin{enumerate}
	\item \textbf{Usuario:} De los usuarios se deben almacenar: un identificador para cada usuario, el nombre completo, el tel\'efono, la direcci\'on, el municipio y la provincia.
	\item \textbf{Deck:} De los decks se deben almacenar: un identificador para cada deck distinto, el nombre del deck, la cantidad de cartas en el mazo principal, en el mazo alternativo y en el mazo extra, el arquetipo y el usuario que lo cre\'o.
	\item \textbf{Torneo:} De los torneos se debe almacenar: un identificador para cada torneo, nombre del torneo, fecha de inicio, direcci\'on y el administrador que lo cre\'o.
	\item \textbf{Rol:} Para cada uno de los roles que puede desempe\~nar un usuario en la aplicaci\'on se guarda un identificador y el nombre del rol.
	\item \textbf{Partida:} De las partidas se debe almacenar: el identificador de cada usuario que participa(2 en total), el identificador del torneo al cual pertenece, la fecha exacta en la cual se realiza, el resultado de la partida(solo hay 4 resultados posibles: "2:1", "2:0", "1:2", "0:2") y la ronda del torneo en la cual se realiza.
	\item \textbf{Solicitud de inscripci\'on en torneos:} Los jugadores solicitan inscribirse en alg\'un torneo disponible y estas solicitudes se almacenan como pendientes hasta que alg\'un administrador las acepte e iscriba al jugador o las rechase. Para esto es necesario almacenar: el identificador del usuario, el identificador del torneo, el identificador del deck con el cual se desea inscribir, la fecha en la cual raliz\'o la solicitud y el estado de la solicitud. Los posibles estados son: pendiente, aceptado, rechazado.
	\item \textbf{Registro de usuario:} Para todos los usuarios se debe almacenar el identificador del usuario relacionado con el identificador de cada rol que este desempe\~na.
\end{enumerate} 
\newpage

\section{Solucion Propuesta}
Hemos optado por la Arquitectura Onion, presentada por Jeffrey Palermo, debido a
su sólido historial de éxito en proyectos de software. Esta elección se basa en los
siguientes motivos clave:
\begin{enumerate}
\item Principio de Inversión de Dependencias: La Arquitectura Onion se adhiere al
Principio de Inversión de Dependencias, promoviendo la independencia entre
capas y la flexibilidad del sistema.
\item Modularidad y Separación de Responsabilidades: Nos permite mantener una
estructura modular y una clara separación de responsabilidades, esenciales
para el desarrollo y mantenimiento eficaz.
\item Testabilidad: Facilita pruebas efectivas, garantizando la calidad y confiabilidad
del código en nuestro proyecto.
\item Flexibilidad y Adaptabilidad: Nos permite adaptarnos a cambios futuros y
nuevas tecnologías sin afectar significativamente la funcionalidad existente.
\item Colaboración Efectiva: Con múltiples ingenieros trabajando en el proyecto, la
Arquitectura Onion establece contratos claros entre capas y componentes, lo
que facilita la colaboración.
\end{enumerate}
\subsection{Capas Externas (Outer Layers)}
\textbf{YuGiOh.Infrastructure:} En esta capa reside nuestra base de datos.\\
%\textbf{YuGiOh.Tests:} En esta capa residen pruebas unitarias, de integración y de extremo a
%extremo esenciales para garantizar la calidad y la confiabilidad del código.\\
\textbf{YuGiOh.API:} Es la capa más cercana a la interfaz de usuario y proporciona una
forma de comunicarse con el núcleo de la aplicación. Esta capa interactúa con la
primera capa del "application core"(application services layer).
\subsection{Capas Internas}
\textbf{YuGiOh.ApplicationServices (Transport Layer):} En esta capa es donde reside la mayor parte de nuestra lógica
de negocios. Lleva a cabo las operaciones para convertir A en B, entrada en salida.
Específicamente en nuestro proyecto, esta capa se encargará de aspectos como la
gestión de emparejamientos, el control de estadísticas, el manejo de algunas reglas
del torneo, la gestión de resultados, el control de decks, entre otros.\\
%Dentro de esta capa, definimos lo que
%nuestro servicio puede hacer a través de una serie de contratos.
\textbf{YuGiOh.ApplicationCore:} En esta capa definimos los crud realizables.
\textbf{YuGiOh.Domain:} Es la capa de representación de los objetos de datos de alto nivel
que utilizamos. Aquí se definen y encapsulan los conceptos fundamentales y los
elementos clave de los torneos, lo que permite una representación fiel y coherente de los
datos en la aplicación. Algunos de los aspectos esenciales que se manejan en esta
capa incluyen la representación de entidades o conceptos fundamentales de la
lógica del proyecto, como jugadores, decks, mazos, partidas y torneos.
\newpage

\section{Resultados Obtenidos}
Vacio pk hacen falta im\'agenes del front.

\newpage

\section{Concusiones y recomendaciones}
Den muela amiguitos.

\newpage

\section{Bibliograf\'ia}
\end{document}